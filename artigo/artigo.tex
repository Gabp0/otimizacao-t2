\documentclass{article}
\usepackage[utf8]{inputenc}
\usepackage[T1]{fontenc}

\title{Otimização - Trabalho 2}
\author{
    Gabriel de Oliveira Pontarolo 
    \texttt{GRR20203895}
    \and 
    Rodrigo Saviam Soffner
    \texttt{GRR20205092}
}
\date{Setembro 2022}

\usepackage{natbib}
\usepackage{graphicx}
\usepackage{amsmath}

\begin{document}

\maketitle

\section{Introdução}
\paragraph{} Esse artigo trata da descrição de uma possível implementação utilizando um algoritmo com a técnica de \emph{branch and bound} para o problema do elenco, o qual consiste em selecionar um determinado subconjunto de atores que atendem a certos requisitos de forma a diminuir o custo total. 

\section{O problema}
\paragraph{} Uma produtora de filmes precisa selecionar um elenco para um filme de forma a minimizar o custo de produção ao mesmo tempo que todos os grupos sociais elencados previamente sejam representados. Temos que:

\begin{itemize}
    \item O conjunto \emph{S} representa os grupos sociais;
    \item O conjunto \emph{A} representa os atores que podem ser selecionados;
    \item O conjunto \emph{P} representa os papéis que podem ser dados aos atores;
    \item Para cada ator \emph{$a \in A$} temos associado um conjunto $\emph{S}_\emph{a} \subseteq \emph{S}$ indiciando os grupos dos quais faz parte e um custo \emph{$v_a$}; 
    \item O subconjunto \emph{$X \subseteq A$} representa os atores selecionados, ou seja, a solução do problema;
    \item Para que a solução seja viável |\emph{X}| = |\emph{P}| ;
    \item Para que a solução seja viável \emph{$\bigcup_{a \in X} S_a = S$};
    \item O valor de \emph{$\sum_{a \in X} v_a$} deve ser mínimo;
\end{itemize}

\section{A modelagem}

\paragraph{} Nessa sessão será feita a análise das decisões feitas em relação a modelagem do algoritmo de \emph{branch and bound}, com ênfase nos aspectos de viabilidade e otimalidade, assim como a geração dos nós da árvore gerada pelas chamadas recursivas.

\paragraph{} Iremos inicializar o conjunto \emph{$X_{opt} \leftarrow \emptyset$} e a variável \emph{$opt \leftarrow \infty$} (pois se trata de um problema de minimização) que representam o conjunto \emph{X} ótimo atual e \emph{$\sum_{a \in X_{opt}} v_a$}, respectivamente. Antes da execução do algoritmo em si, o conjunto \emph{A}, implementado utilizando um vetor, foi ordenado de acordo com os valores \emph{$v_a$}, em ordem crescente, para ser utilizado futuramente na função de \emph{bound}.

\paragraph{} Assim, para saber se atingiu um nó folha e a solução é viável, o algoritmo executa o teste |\emph{X}| = |\emph{P}| e \emph{$\bigcup_{a \in X} S_a = S$}. Em caso verdadeiro, se \emph{$\sum_{a \in X} v_a \leq opt$}, então \emph{$opt \leftarrow \sum_{a \in X} v_a$} e \emph{$X_{opt} \leftarrow X$}. Note que, no primeiro nó folha viável, o teste esse teste sempre será verdadeiro. Também é possível que ele atinja um nó folha quando |\emph{X}| = |\emph{P}| e \emph{$\bigcup_{a \in X} S_a = S$} é falso, porém \emph{A} = $\emptyset$ é verdadeiro. Nesse caso, a solução é inviável.

\paragraph{} Se o algoritmo não atingiu um nó folha, ele irá testar se ainda é possível gerar uma solução viável naquele ramo da árvore, ou seja, se |\emph{X}| + |\emph{A}| $\geq$ |\emph{P}| e \emph{$(\bigcup_{a \in X} S_a) \cup (\bigcup_{a \in A} S_a) = S$}. Caso seja falso, tal ramo da árvore será "cortado", ou seja, não irá executar a chamada recursiva.

\paragraph{} O último teste executado antes da chamada recursiva utiliza uma função de \emph{bound}, ou limitante, para saber se ainda é possível atingir um valor melhor que o ótimo naquele ramo. Note que, para que haja um ganho no algoritmo, a função escolhida para calcular o limitante precisa ser mais eficiente do que a "descida" até a folha do ramo atual da árvore. Entretanto, o valor retornado não precisa ser exato, pois servirá apenas como um limite superior, mas deve ser "otimista", retornado um valor melhor do que aquele que seria obtido no ramo. Nesse caso, sendo este um problema de minimização, temos que se \emph{bound(X, A)} $\leq$ \emph{opt} é falso, a chamada recursiva não será executada. Será feita a análise da função escolhida na próxima sessão desse artigo.

\section{O programa linear}

\end{document}