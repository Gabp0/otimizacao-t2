\documentclass{article}
\usepackage[utf8]{inputenc}

\title{Otimização - Trabalho 2}
\author{
    Gabriel de Oliveira Pontarolo 
    \texttt{GRR20203895}
    \and 
    Rodrigo Saviam Soffner
    \texttt{GRR20205092}
}
\date{Setembro 2022}

\usepackage{natbib}
\usepackage{graphicx}
\usepackage{amsmath}

\begin{document}

\maketitle

\section{Introdução}
\paragraph{} Esse artigo trata da descrição de uma possível implementação utilizando um algoritmo com a técnica de \emph{branch and bound} para o problema do elenco, o qual consiste em selecionar um determinado subconjunto de atores que atendem a certos requisitos de forma a diminuir o custo total. 

\section{O problema}
\paragraph{} Uma usina hidroelétrica e uma termoelétrica, a qual só é utilizada quando a primeira não é suficiente para suprimir a demanda, abastecem a rede de energia de uma cidade. O objetivo do problema é conceber um plano de geração mensal, durante um período de \emph{n} meses que minimiza o custo considerando as seguintes informações:

\begin{itemize}
    \item O custo de geração de hidroelétrica é nulo, porém ela possui um custo ambiental em função da variação do volume, em $\emph{m}^3$, do reservatório de um mês para outro;
    \item A termoelétrica é associado um custo de geração para cada \emph{MWatt} de energia;
    \item A termoelétrica possui um limite de geração mensal;
    \item O reservatório possui um limite mínimo e máximo em $\emph{m}^3$;
    \item O reservatório começa com um volume inicial em $\emph{m}^3$;
    \item O reservatório recebe um volume de água mensal ($\emph{m}^3$) proveniente de chuvas, afluências, etc;
    \item A cada 1 $\emph{m}^3$ de água consumido do reservatório é gerado \emph{kMWatt} de energia, onde \emph{k} é um coeficiente de geração dado;
    \item A geração das duas usinas juntas deve suprir as demandas mensais da cidade. Não é um problema gerar mais do que o necessário;
\end{itemize}

\section{A modelagem}
\paragraph{} Incialmente, vamos destacar os valores dados:
\begin{itemize}
    \item \emph{CA} é o custo ambiental da hidroelétrica pela variação de volume por $\emph{m}^3$;
    \item \emph{CT} é o custo de geração da termoelétrica por \emph{MWatt};
    \item $\emph{t}_{max}$ é a geração máxima da termoelétrica em \emph{MWatt};
    \item $\emph{v}_{ini}$ é o volume inicial do reservatório;
    \item $\emph{v}_{min}$ é o volume mínimo do reservatório e $\emph{v}_{max}$ é o volume máximo;
    \item $\emph{y}_1$,$\emph{y}_2$,\dots,$\emph{y}_n$ consistem da afluência recebida pelo reservatório entre os meses \emph{1} até \emph{n};
    \item \emph{k} é a constante de geração do reservatório por $\emph{m}^3$;
    \item $\emph{d}_1$,$\emph{d}_2$,\dots,$\emph{d}_n$ consistem da demanda de energia da cidade entre os meses \emph{1} até \emph{n};
\end{itemize}
\paragraph{} Para os valores que serão calculados temos:
\begin{itemize}
    \item $\emph{h}_1$,$\emph{h}_2$,\dots,$\emph{h}_n$ consistem do consumo de água da hidroelétrica entre os meses \emph{1} até \emph{n};
    \item $\emph{t}_1$,$\emph{t}_2$,\dots,$\emph{t}_n$ consistem da geração de energia da termoelétrica entre os meses \emph{1} até \emph{n};
    \item $\emph{v}_1$,$\emph{v}_2$,\dots,$\emph{v}_n$ consistem do volume do reservatório entre os meses \emph{1} até \emph{n};
\end{itemize}
\paragraph{} Desse modo, a função objetivo que queremos minimizar é \emph{$\sum_{i=1}^{n} (CA \cdot | y_{i} - h_{i} | + CT \cdot t_{i})$}, onde temos que \emph{$|y_{i} - h_{i}|$} é a variação do volume do reservatório no final do mês, a qual é multiplicada pelo custo ambiental, e $CT \cdot t_{i}$ representa o custo de geração da termoelétrica. Ambos mês a mês de \emph{1} até \emph{n}.
\paragraph{} É importante notar que, como se trata de um programa linear, é preciso substituir o modulo pois esse não é uma função linear. Assim, será introduzido as váriaveis positivas $\emph{x}_1$,$\emph{x}_2$,\dots,$\emph{x}_n$ que representam o aumento do volume e $\emph{z}_1$,$\emph{z}_2$,\dots,$\emph{z}_n$ para representar a diminuição. \emph{$| y_{i} - h_{i} |$} será substituído por \emph{$(x_i + z_i)$} e serão adicionadas as restrições \emph{$x_i - z_i = y_i - h_i$} e \emph{$x_i,z_i \ge 0$} para manter a igualdade. A função objetivo final será \emph{$\sum_{i=1}^{n} (CA \cdot (x_i + z_i) + CT \cdot t_{i})$};
\paragraph{} Temos também que, o volume do mês 1 consiste de \emph{$v_1 = v_{ini} + y_1$}, e o volume dos meses subsequentes de \emph{$v_j = v_{j-1} + y_j - h_{j-1}$} onde temos o volume do mês anterior somado a afluência e subtraido do consumo do mês anterior. 
\paragraph{} Ademais, o consumo de água do mês atual não pode ser maior do que o volume, assim \emph{$h_i \le v_i$} e o volume em cada um dos meses precisa estar entre o mínimo e o máximo, dessa forma \emph{$v_{min} \le v_i - h_i \le v_{max}$}.
\paragraph{} Finalmente, temos as restrições de demanda \emph{$k \cdot h_i + t_i \ge d_i $}, onde a soma da geração das duas usinas precisa ser maior ou igual a demanda mensal.

\section{O programa linear}
\paragraph{} O programa linear resultante será:

\begin{equation*}
    \begin{array}{ll@{}ll}
        \textbf{min}  & \sum_{i=1}^{n} (CA \cdot x_i + CA \cdot z_i + CT \cdot t_{i}) &\\
        \textbf{s.t.} & x_i - z_i = y_i - h_i &i=1 ,\dots, n\\
                      & v_1 = v_{ini} + y_1 &\\ 
                      & v_j = v_{j-1} + y_j - h_{j-1} &j=2 ,\dots, n-1&\\
                      & h_i \le v_i &\\
                      & v_{min} \le v_i - h_i \le v_{max} &\\
                      & k \cdot h_i + t_i \ge d_i &\\
                      & x_i,z_i,h_i,t_i \ge 0
    \end{array}
\end{equation*}


\end{document}